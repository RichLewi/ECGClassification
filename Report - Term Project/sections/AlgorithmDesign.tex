\section{Algorithm Design Using Spiral Approach}
The algorithm design for the ECG heartbeat classification system is based on the spiral approach, which involves iterative development and refinement of the system. 

Based on gained knowledge from the course and literature such as in \cite{b3}, \cite{b4} and \cite{b5} on ECG classification two algorithms were chosen to be examined on their performance on ECG heartbeat classification: Convolutional Neural Networks (CNN) and Recurrent Neural Networks (RNNs). CNNs automatically learn and extract relevant features from raw ECG signals, by recognizing characteristic wave patters. Limitations of CNNs are their limitation to a fixed receptive field, they might overlook long-term dependencies in the heartbeat sequence. 
Recurrent neural networks like LSTMs (Long Short-Term Memory) and GRUs (Gated Recurrent Units) are designed to capture temporal dependencies in time-series data such as ECG. LSTMs maintain memory cells with gated updates, allowing them to retain information over extended periods. Notably, they are designed to overcome the vanishing gradient problem in RNNs. GRUs are simplified RNN variant with only update and reset gates, which makes them computationally more efficient than LSTMs.

Combining CNNs with LSTMs or GRU leverages the strengths of both approaches. In these hybrids, the CNN layers first act as feature extractors that convert raw ECG signals into high-level feature sequences, which are then fed into LSTM or GRU layers that model the temporal dynamics. These architectures are effective because they can learn subtle ECG features and also the timing irregularities that signal different arrhythmias. 

\subsection{Model Design for Iteration 1}
The CNN model designed for classification in this project consist of three Convolutional Layers with 32, 64 and 128 filters, kernel size of 3x3, and ReLU activation function. A MaxPooling Layer is used to downsample the feature maps. Before feeding the outputs to a number of Dense Layers a Flatten Layer is used to convert the 2D feature maps to a 1D feature vector. For the Dense layers we use a structure of 64 and 32 neurons with ReLU activation function. The output layer consists of five neurons, one for each class, and uses the softmax activation function.

The CNN-LSTM model consists of a CNN model with three Convolutional Layers and a MaxPooling Layer, and then fed into the LSTM layer. The LSTM layer consists of 64 units. Its output is than flattened and fed into two Dense Layers with 64 and 32 neurons utilizing ReLU activation function. In between the Dense layers a Dropout layer with dropout rate 0.25 is used to prevent overfitting.
The output layer stays the same, it has 5 neurons.

For the CNN-GRU model the same architecture is used, but instead of the LSTM layer a GRU layer with 65 units is used.

The models are trained using the Adam optimizer with a learning rate of 0.001, a batch size of 64, and a categorical cross-entropy loss function. The models are trained for 20 epochs using batch size 32. Callbacks have been used to save the best model based on the validation accuracy and early stopping with a patience of 5 epochs to prevent overfitting and reduce training time. The learning rate is reduced by a factor of 0.5 if the validation loss does not improve for 3 epochs.
The training data is split into 80\% training and 20\% validation data. The model with the best validation accuracy is saved and evaluated on the test dataset.

\subsection{Model Evaluation for Iteration 1}
The models are evaluated based on their performance metrics on the test set. The performance metrics include accuracy, precision, recall, and F1-score. The confusion matrices are used to evaluate the models' performance on each class. The training time of the models is also considered as a performance metric.
In table \ref{tab:model_performance} the performance metrics of the models on the test set are compared. The CNN and CNN-GRU model basically have the same performance, their accuracy, precision, recall, and F1-score are 98.36\%. The CNN-LSTM model has an accuracy of 98.20\%. Looking at the confusion matrices in table \ref{tab:confusion_matrices} and \ref{tab:normalized_conf_percent}, one can see that classes 1 (Supraventricular) and 3 (Fusion) are the most difficult to classify. The CNN-LSTM model has the best performance on class 1, while the CNN model has the best performance on class 3 classes. The CNN-LSTM model has the wors performance on these difficult classes but performs well on the other classes. From this we can not conclude which model performs best, but the CNN and CNN-GRU model have the same performance and the CNN-LSTM model has a slightly worse performance. Important to note is that the CNN model has the significantly shortest training time of 8minutes, while the CNN-LSTM model has the longest training time of 94 minutes. The CNN-GRU model has a training time of 74 minutes. The CNN model is the most efficient model in terms of training time and performance.

Next we try to improve the performance of the models especially on class 1 and 3 by changing the architecture.

\begin{table}[bt]
    \centering
    \caption{Comparison of Model Performance Metrics (Iteration 1)}
    \label{tab:model_performance}
    \begin{tabular}{lcccccc}
        \toprule
        \textbf{Model} & \textbf{Training Time} & \textbf{Acc.} & \textbf{Prec.} & \textbf{Rec.} & \textbf{F1} \\
        \midrule
        CNN\_It\_1     & 8min 4.4s  & 98.36\% & 98.36\% & 98.36\% & 98.36\% \\
        CNN\_LSTM\_It\_1 & 94min 4.9s & 98.20\% & 98.27\% & 98.20\% & 98.23\% \\
        CNN\_GRU\_It\_1  & 74min 33.7s & 98.36\% & 98.36\% & 98.36\% & 98.36\% \\
        \bottomrule
    \end{tabular}
\end{table}

\begin{table}[ht]
    \centering
    \caption{Comparison of Confusion Matrices for Iteration 1 Models}
    \label{tab:confusion_matrices}
    \begin{minipage}{0.32\textwidth}
        \centering
        \caption*{CNN\_Model\_Iteration\_1}
        \begin{tabular}{cccccc}
            \toprule
            & \multicolumn{5}{c}{Predicted} \\
            \cmidrule(lr){2-6}
            Actual & 0 & 1 & 2 & 3 & 4 \\
            \midrule
            0 & 17957 & 81  & 46  & 17  & 16 \\
            1 & 95    & 453 & 7   & 0   & 1  \\
            2 & 36    & 1   & 1392& 18  & 1  \\
            3 & 13    & 0   & 12  & 137 & 0  \\
            4 & 12    & 1   & 1   & 0   & 1594 \\
            \bottomrule
        \end{tabular}
    \end{minipage}
    \hfill
    \begin{minipage}{0.32\textwidth}
        \centering
        \caption*{CNN\_LSTM\_Model\_Iteration\_1}
        \begin{tabular}{cccccc}
            \toprule
            & \multicolumn{5}{c}{Predicted} \\
            \cmidrule(lr){2-6}
            Actual & 0 & 1 & 2 & 3 & 4 \\
            \midrule
            0 & 17913 & 136 & 32  & 12  & 24 \\
            1 & 74    & 474 & 4   & 3   & 1  \\
            2 & 33    & 5   & 1383& 23  & 4  \\
            3 & 15    & 0   & 12  & 135 & 0  \\
            4 & 13    & 2   & 1   & 0   & 1592 \\
            \bottomrule
        \end{tabular}
    \end{minipage}
    \hfill
    \begin{minipage}{0.32\textwidth}
        \centering
        \caption*{CNN\_GRU\_Model\_Iteration\_1}
        \begin{tabular}{cccccc}
            \toprule
            & \multicolumn{5}{c}{Predicted} \\
            \cmidrule(lr){2-6}
            Actual & 0 & 1 & 2 & 3 & 4 \\
            \midrule
            0 & 17948 & 81  & 40  & 19  & 29 \\
            1 & 77    & 466 & 11  & 0   & 2  \\
            2 & 28    & 6   & 1389& 19  & 6  \\
            3 & 15    & 1   & 17  & 129 & 0  \\
            4 & 7     & 0   & 2   & 0   & 1599 \\
            \bottomrule
        \end{tabular}
    \end{minipage}
\end{table}

\begin{table}[ht]
    \centering
    \caption{Normalized Confusion Matrices in Percent}
    \label{tab:normalized_conf_percent}
    % First Model: CNN_Model_Iteration_1
    \begin{minipage}{0.32\textwidth}
        \centering
        \caption*{CNN\_Model\_Iteration\_1}
        \begin{tabular}{cccccc}
            \toprule
            & \multicolumn{5}{c}{Predicted} \\
            \cmidrule(lr){2-6}
            Actual & 0 & 1 & 2 & 3 & 4 \\
            \midrule
            0 & 99.12\% & 0.45\% & 0.25\% & 0.09\% & 0.09\% \\
            1 & 17.09\% & 81.47\% & 1.26\% & 0.00\% & 0.18\% \\
            2 & 2.49\%  & 0.07\% & 96.13\% & 1.24\% & 0.07\% \\
            3 & 8.02\%  & 0.00\% & 7.41\%  & 84.57\% & 0.00\% \\
            4 & 0.75\%  & 0.06\% & 0.06\%  & 0.00\% & 99.13\% \\
            \bottomrule
        \end{tabular}
    \end{minipage}
    \hfill
    % Second Model: CNN_LSTM_Model_Iteration_1
    \begin{minipage}{0.32\textwidth}
        \centering
        \caption*{CNN\_LSTM\_Model\_Iteration\_1}
        \begin{tabular}{cccccc}
            \toprule
            & \multicolumn{5}{c}{Predicted} \\
            \cmidrule(lr){2-6}
            Actual & 0 & 1 & 2 & 3 & 4 \\
            \midrule
            0 & 98.87\% & 0.75\% & 0.18\% & 0.07\% & 0.13\% \\
            1 & 13.31\% & 85.25\% & 0.72\% & 0.54\% & 0.18\% \\
            2 & 2.28\%  & 0.35\% & 95.51\% & 1.59\% & 0.28\% \\
            3 & 9.26\%  & 0.00\% & 7.41\%  & 83.33\% & 0.00\% \\
            4 & 0.81\%  & 0.12\% & 0.06\%  & 0.00\% & 99.00\% \\
            \bottomrule
        \end{tabular}
    \end{minipage}
    \hfill
    % Third Model: CNN_GRU_Model_Iteration_1
    \begin{minipage}{0.32\textwidth}
        \centering
        \caption*{CNN\_GRU\_Model\_Iteration\_1}
        \begin{tabular}{cccccc}
            \toprule
            & \multicolumn{5}{c}{Predicted} \\
            \cmidrule(lr){2-6}
            Actual & 0 & 1 & 2 & 3 & 4 \\
            \midrule
            0 & 99.07\% & 0.45\% & 0.22\% & 0.10\% & 0.16\% \\
            1 & 13.85\% & 83.81\% & 1.98\% & 0.00\% & 0.36\% \\
            2 & 1.93\%  & 0.41\% & 95.93\% & 1.31\% & 0.41\% \\
            3 & 9.26\%  & 0.62\% & 10.49\% & 79.63\% & 0.00\% \\
            4 & 0.44\%  & 0.00\% & 0.12\% & 0.00\% & 99.44\% \\
            \bottomrule
        \end{tabular}
    \end{minipage}
\end{table}

\subsection{Model Design for Iteration 2}
In the second iteration, the models are improved by adding residual connections to the CNN layers. Residual connections are used to prevent the vanishing gradient problem and allow for deeper networks. The CNN parts of all the models are extended by adding residual connections between the convolutional layers. The models are trained using the same hyperparameters as in the first iteration. 

\subsection{Model Evaluation for Iteration 2}

The performance metrics of the refined models are shown in table \ref{tab:eval_metrics_it2}. The CNN model has an accuracy of 98.15\%, the CNN-LSTM model has an accuracy of 98.57\%, and the CNN-GRU model has an accuracy of 97.88\%. The refined CNN model and CNN-GRU model have not improved the performance. The refined CNN-LSTM model has a slightly better performance. 

\begin{table}[hbt]
    \centering
    \caption{Performance Metrics for Refined Iteration 2 Models}
    \label{tab:eval_metrics_it2}
    \begin{tabular}{lcccccc}
        \toprule
        \textbf{Model} & \textbf{Training Time} & \textbf{Accuracy} & \textbf{Prec.} & \textbf{Rec.} & \textbf{F1} \\
        \midrule
        CNN\_It\_1     & 8min 4.4s  & 98.36\% & 98.36\% & 98.36\% & 98.36\% \\
        CNN\_LSTM\_It\_1 & 94min 4.9s & 98.20\% & 98.27\% & 98.20\% & 98.23\% \\
        CNN\_GRU\_It\_1  & 74min 33.7s & 98.36\% & 98.36\% & 98.36\% & 98.36\% \\

        CNN\_It\_2      & 120min 41.4s & 98.15\% & 98.20\% & 98.21\% & 98.21\% \\
        CNN\_LSTM\_It\_2 & 259min 55.6s & 98.57\% & 98.56\% & 98.57\% & 98.56\% \\
        CNN\_GRU\_It\_2  & 113min 9.3s & 97.88\% & 97.93\% & 97.88\% & 97.90\% \\
        \bottomrule
    \end{tabular}
\end{table}

Following the confusion matrices for the models of the second iteration are shown. The confusion matrices with absolute values are shown in table \ref{tab:conf_abs_it2} and the normalized confusion matrices in percent are shown in table \ref{tab:conf_matrices_it2}. The refined CNN-LSTM model has a slightly better performance on all classes apart from the Supraventricular class. The models all have problems finding distinct features between the Supraventricular and Normal classes, aswell as finding the differences from the Fusion class to both the Normal and Ventricular classes. 

\begin{table}[ht]
    \centering
    \caption{Confusion Matrices with Absolute Values (Iteration 2)}
    \label{tab:conf_abs_it2}
    \begin{minipage}{0.32\textwidth}
        \caption*{CNN\_Model\_Iteration\_2}
        \begin{tabular}{cccccc}
            \toprule
            & \multicolumn{5}{c}{Predicted} \\
            \cmidrule(lr){2-6}
            Actual & 0 & 1 & 2 & 3 & 4 \\
            \midrule
            0 & 17900 & 100 & 75  & 22  & 20  \\
            1 & 82    & 462 & 11  & 1   & 0   \\
            2 & 24    & 1   & 1395& 24  & 4   \\
            3 & 9     & 2   & 14  & 137 & 0   \\
            4 & 13    & 1   & 2   & 1   & 1591\\
            \bottomrule
        \end{tabular}
    \end{minipage}
    \hfill
    \begin{minipage}{0.32\textwidth}
        \caption*{CNN\_LSTM\_Model\_Iteration\_2}
        \begin{tabular}{cccccc}
            \toprule
            & \multicolumn{5}{c}{Predicted} \\
            \cmidrule(lr){2-6}
            Actual & 0 & 1 & 2 & 3 & 4 \\
            \midrule
            0 & 18000 & 51  & 36  & 16  & 14  \\
            1 & 89    & 455 & 9   & 2   & 1   \\
            2 & 30    & 4   & 1393& 20  & 1   \\
            3 & 12    & 1   & 12  & 137 & 0   \\
            4 & 14    & 0   & 1   & 0   & 1593\\
            \bottomrule
        \end{tabular}
    \end{minipage}
    \hfill
    \begin{minipage}{0.32\textwidth}
        \caption*{CNN\_GRU\_Model\_Iteration\_2}
        \begin{tabular}{cccccc}
            \toprule
            & \multicolumn{5}{c}{Predicted} \\
            \cmidrule(lr){2-6}
            Actual & 0 & 1 & 2 & 3 & 4 \\
            \midrule
            0 & 17880 & 114 & 83  & 17  & 23  \\
            1 & 103   & 444 & 7   & 2   & 0   \\
            2 & 23    & 10  & 1388& 20  & 7   \\
            3 & 13    & 1   & 17  & 131 & 0   \\
            4 & 13    & 3   & 7   & 0   & 1585\\
            \bottomrule
        \end{tabular}
    \end{minipage}
\end{table}

\vspace{1em}

\begin{table}[ht]
    \centering
    \caption{Normalized Confusion Matrices in Percent (Iteration 2)}
    \label{tab:conf_matrices_it2}
    \begin{minipage}{0.32\textwidth}
        \caption*{CNN\_Model\_Iteration\_2}
        \begin{tabular}{cccccc}
            \toprule
            & \multicolumn{5}{c}{Predicted} \\
            \cmidrule(lr){2-6}
            Actual & 0 & 1 & 2 & 3 & 4 \\
            \midrule
            0 & 98.80\% & 0.55\% & 0.41\% & 0.12\% & 0.11\% \\
            1 & 14.75\% & 83.09\% & 1.98\% & 0.18\% & 0.00\% \\
            2 & 1.66\%  & 0.07\% & 96.34\% & 1.66\% & 0.28\% \\
            3 & 5.56\%  & 1.23\% & 8.64\%  & 84.57\% & 0.00\% \\
            4 & 0.81\%  & 0.06\% & 0.12\%  & 0.06\% & 98.94\% \\
            \bottomrule
        \end{tabular}
    \end{minipage}
    \hfill
    \begin{minipage}{0.32\textwidth}
        \caption*{CNN\_LSTM\_Model\_Iteration\_2}
        \begin{tabular}{cccccc}
            \toprule
            & \multicolumn{5}{c}{Predicted} \\
            \cmidrule(lr){2-6}
            Actual & 0 & 1 & 2 & 3 & 4 \\
            \midrule
            0 & 99.35\% & 0.28\% & 0.20\% & 0.09\% & 0.08\% \\
            1 & 16.01\% & 81.83\% & 1.62\% & 0.36\% & 0.18\% \\
            2 & 2.07\%  & 0.28\% & 96.20\% & 1.38\% & 0.07\% \\
            3 & 7.41\%  & 0.62\% & 7.41\%  & 84.57\% & 0.00\% \\
            4 & 0.87\%  & 0.00\% & 0.06\%  & 0.00\% & 99.07\% \\
            \bottomrule
        \end{tabular}
    \end{minipage}
    \hfill
    \begin{minipage}{0.32\textwidth}
        \caption*{CNN\_GRU\_Model\_Iteration\_2}
        \begin{tabular}{cccccc}
            \toprule
            & \multicolumn{5}{c}{Predicted} \\
            \cmidrule(lr){2-6}
            Actual & 0 & 1 & 2 & 3 & 4 \\
            \midrule
            0 & 98.69\% & 0.63\% & 0.46\% & 0.09\% & 0.13\% \\
            1 & 18.53\% & 79.86\% & 1.26\% & 0.36\% & 0.00\% \\
            2 & 1.59\%  & 0.69\% & 95.86\% & 1.38\% & 0.48\% \\
            3 & 8.02\%  & 0.62\% & 10.49\% & 80.86\% & 0.00\% \\
            4 & 0.81\%  & 0.19\% & 0.44\% & 0.00\% & 98.57\% \\
            \bottomrule
        \end{tabular}
    \end{minipage}
\end{table}

This change did increase training time significantly without leading to really noticable improvements, only the LSTM model had some improvements. Therefor it was chosen to revert all the models to the original architecture and make more adjustments from there for iteration 3. 

\subsection{Model Design for Iteration 3}
In the third iteration, the models are improved by adding batch normalization layers after each convolutional layer. Batch normalization layers are used to normalize the activations of the previous layer at each batch. This helps to stabilize and speed up the training process. 
For both CNN-LSTM and CNN-GRU models, the LSTM and GRU layers are extended by introducing bidirectional layers. Bidirectional layers process the input sequence in both forward and backward directions, capturing dependencies in both directions. Instead of flattening the entire sequence output from the recurrent layers, global average pooling is used to reduce the sequence to a single vector. The global average pooling computes the average over the time dimension as it often leads to better generalization. An additional dropout layer is added after the last dense layer to prevent overfitting.
The models are trained using the same hyperparameters as in the first iteration.

\subsection{Model Evaluation for Iteration 3}
The performance metrics of the refined models are shown in table \ref{tab:eval_metrics_it3}. The CNN model has an accuracy of 97.88\%, the CNN-LSTM model has an accuracy of 98.20\%, and the CNN-GRU model has an accuracy of 98.04\%. The refined models have not improved the performance. Precision, recall and f1-score are also lower than in the first iteration.

\begin{table}[hbt]
    \centering
    \caption{Performance Metrics for Refined Iteration 3 Models}
    \label{tab:eval_metrics_it3}
    \begin{tabular}{lcccccc}
        \toprule
        \textbf{Model} & \textbf{Training Time} & \textbf{Accuracy} & \textbf{Prec.} & \textbf{Rec.} & \textbf{F1} \\
        \midrule
        CNN\_It\_1     & 8min 4.4s  & 98.36\% & 98.36\% & 98.36\% & 98.36\% \\
        CNN\_LSTM\_It\_1 & 94min 4.9s & 98.20\% & 98.27\% & 98.20\% & 98.23\% \\
        CNN\_GRU\_It\_1  & 74min 33.7s & 98.36\% & 98.36\% & 98.36\% & 98.36\% \\

        CNN\_It\_3      & 34min 33.9s & 97.88\% & 89.01\% & 90.72\% & 89.82\% \\
        CNN\_LSTM\_It\_3 & 28min 20.7s & 98.20\% & 89.67\% & 91.90\% & 90.75\% \\
        CNN\_GRU\_It\_3  & 30min 50.9s & 98.04\% & 87.82\% & 92.90\% & 90.14\% \\
        \bottomrule
    \end{tabular}
\end{table}

The confusion matrices in table \ref{tab:conf_abs_it3} and \ref{tab:conf_matrices_it3} show the models performance on the classification task of the test set. The CNN models correct classification rate on class 1 increased by 1\%, but it missclassifies 6\% more of class 3. The CNN-LSTM models performs slightly worse on all classes. The changes only had a postive effect on the CNN-GRU model, as it has increased its performance on class 3 from before 79.63\% to now 87.04\%. It also does not perform significantly worse on any other class. The training time of the models is between 28 and 34 minutes, with the CNN-LSTM model having the shortest training time.

\begin{table}[ht]
    \centering
    \caption{Confusion Matrices with Absolute Values}
    \label{tab:conf_abs_it3}
    \begin{minipage}{0.32\textwidth}
        %\centering
        \caption*{CNN\_Model\_Iteration\_3}
        \begin{tabular}{cccccc}
            \toprule
            & \multicolumn{5}{c}{Predicted} \\
            \cmidrule(lr){2-6}
            Actual & 0 & 1 & 2 & 3 & 4 \\
            \midrule
            0 & 17864 & 147 & 69  & 10  & 27  \\
            1 & 87    & 459 & 8   & 2   & 0   \\
            2 & 26    & 6   & 1390& 19  & 7   \\
            3 & 16    & 0   & 20  & 126 & 0   \\
            4 & 13    & 5   & 3   & 0   & 1587\\
            \bottomrule
        \end{tabular}
    \end{minipage}
    \hfill
    \begin{minipage}{0.32\textwidth}
        %\centering
        \caption*{CNN\_LSTM\_Model\_Iteration\_3}
        \begin{tabular}{cccccc}
            \toprule
            & \multicolumn{5}{c}{Predicted} \\
            \cmidrule(lr){2-6}
            Actual & 0 & 1 & 2 & 3 & 4 \\
            \midrule
            0 & 17914 & 116 & 44  & 20  & 23  \\
            1 & 85    & 458 & 10  & 3   & 0   \\
            2 & 33    & 3   & 1392& 18  & 2   \\
            3 & 15    & 0   & 13  & 134 & 0   \\
            4 & 6     & 1   & 2   & 1   & 1598\\
            \bottomrule
        \end{tabular}
    \end{minipage}
    \hfill
    \begin{minipage}{0.32\textwidth}
        %\centering
        \caption*{CNN\_GRU\_Model\_Iteration\_3}
        \begin{tabular}{cccccc}
            \toprule
            & \multicolumn{5}{c}{Predicted} \\
            \cmidrule(lr){2-6}
            Actual & 0 & 1 & 2 & 3 & 4 \\
            \midrule
            0 & 17872 & 122 & 65  & 34  & 24  \\
            1 & 76    & 466 & 10  & 2   & 2   \\
            2 & 27    & 6   & 1388& 23  & 4   \\
            3 & 6     & 1   & 14  & 141 & 0   \\
            4 & 9     & 0   & 2   & 3   & 1594\\
            \bottomrule
        \end{tabular}
    \end{minipage}
\end{table}

\vspace{1em}

\begin{table}[ht]
    \centering
    \caption{Normalized Confusion Matrices in Percent}
    \label{tab:conf_matrices_it3}
    % CNN_Model_Iteration_3
    \begin{minipage}{0.32\textwidth}
        %\centering
        \caption*{CNN\_Model\_Iteration\_3}
        \begin{tabular}{cccccc}
            \toprule
            & \multicolumn{5}{c}{Predicted} \\
            \cmidrule(lr){2-6}
            Actual & 0 & 1 & 2 & 3 & 4 \\
            \midrule
            0 & 98.60\% & 0.81\% & 0.38\% & 0.06\% & 0.15\% \\
            1 & 15.65\% & 82.55\% & 1.44\% & 0.36\% & 0.00\% \\
            2 & 1.80\%  & 0.41\% & 95.99\% & 1.31\% & 0.48\% \\
            3 & 9.88\%  & 0.00\% & 12.35\%& 77.78\% & 0.00\% \\
            4 & 0.81\%  & 0.31\% & 1.87\% & 0.00\% & 98.69\% \\
            \bottomrule
        \end{tabular}
    \end{minipage}
    \hfill
    % CNN_LSTM_Model_Iteration_3
    \begin{minipage}{0.32\textwidth}
        %\centering
        \caption*{CNN\_LSTM\_Model\_Iteration\_3}
        \begin{tabular}{cccccc}
            \toprule
            & \multicolumn{5}{c}{Predicted} \\
            \cmidrule(lr){2-6}
            Actual & 0 & 1 & 2 & 3 & 4 \\
            \midrule
            0 & 98.88\% & 0.64\% & 0.24\% & 0.11\% & 0.13\% \\
            1 & 15.29\% & 82.37\% & 1.80\% & 0.54\% & 0.00\% \\
            2 & 2.28\%  & 0.21\% & 96.13\% & 1.24\% & 0.14\% \\
            3 & 9.26\%  & 0.00\% & 8.02\%  & 82.72\% & 0.00\% \\
            4 & 0.37\%  & 0.62\% & 0.12\%  & 0.62\% & 99.38\% \\
            \bottomrule
        \end{tabular}
    \end{minipage}
    \hfill
    % CNN_GRU_Model_Iteration_3
    \begin{minipage}{0.32\textwidth}
        %\centering
        \caption*{CNN\_GRU\_Model\_Iteration\_3}
        \begin{tabular}{cccccc}
            \toprule
            & \multicolumn{5}{c}{Predicted} \\
            \cmidrule(lr){2-6}
            Actual & 0 & 1 & 2 & 3 & 4 \\
            \midrule
            0 & 98.65\% & 0.67\% & 0.36\% & 0.19\% & 0.13\% \\
            1 & 13.67\% & 83.81\% & 1.80\% & 0.36\% & 0.36\% \\
            2 & 1.86\%  & 0.41\% & 95.86\% & 1.59\% & 0.28\% \\
            3 & 3.70\%  & 0.62\% & 8.64\%  & 87.04\% & 0.00\% \\
            4 & 0.56\%  & 0.00\% & 0.12\%  & 0.19\% & 99.13\% \\
            \bottomrule
        \end{tabular}
    \end{minipage}
\end{table}

The problems in classification of classes 1 (Supraventricular) and 3 (Fusion) are still present. The models are not able to classify these classes correctly, probably due to their low number of samples in the used dataset. Also, looking at the random examples of each classes heartbeats, for the untrained eye the differnces between the classes are not clear. 
The models are not able to learn the features of these classes well enough, so they can not generalize well on these classes, which is a problem for real-world applications. The models are not able to detect these arrhythmias, which can be dangerous for patients. The models need to be improved to be able to classify these classes correctly. 