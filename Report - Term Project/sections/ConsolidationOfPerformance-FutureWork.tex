\section{Consolidation of Performance and Future Plan}
This section consolidates the performance of our ECG classification models and outlines suggestions for future work to enhance model performance and address the limitations observed in the current implementation.

\subsection{Consolidation of Performance}
The overall performance of our ECG classification was evaluated using multiple metrics, including accuracy, precision, recall and F1-score. The models trained in Iteration 1 already had really good performance, the best with an accuracy of 98.76\% with a training time of only 8minutes.
However, a closer look at the per-class performance revealed that certain classes—especially classes 1 and 3 were more challenging. The performance disparities in these classes suggest that while the models are robust overall, the discriminative features for these underperforming classes are not being captured optimally. This could be due to the class imbalance in the dataset, which may have led to the models being biased towards the majority classes. To address this issue, we suggest to experiment with different data augmentation techniques, such as SMOTE, ADASYN instead of simply utilizing Random Oversampling, to balance the dataset and improve the models' performance on the underrepresented classes. 
Moreover, training times were within acceptable limits (ranging from approximately 8 to 100 minutes per model), indicating that the proposed architectures are computationally efficient for the current dataset.

Our ECG classification models were developed and evaluated based on the project specifications metioned in the Requirement and Data Analysis section \ref{sec:project-specs}. The models (CNN, CNN-LSTM, and CNN-GRU) achieved overall accuracies well above the 95\% target and completed training in under 100 minutes, far below the five hour limit. However, while the overall metrics are satisfactory, the per-class performance, especially for arrhythmia classes, suggests room for improvement in recall and precision.

Table~\ref{tab:current_specs} below summarizes the current specifications alongside our achieved results and comments for further improvement:

\begin{table}[b]
    \centering
    \caption{Current Project Specifications vs. Achieved Performance}
    \label{tab:current_specs}
    \begin{tabular}{|p{4.2cm}|p{4.0cm}|p{1.7cm}|p{5.2cm}|}
        \hline
        \textbf{Specification} & \textbf{Requirement} & \textbf{Met?} & \textbf{Comments / Actions} \\
        \hline
        Dataset & MIT-BIH Arrhythmia (5 classes) & Yes & Standard dataset used as specified. \\
        \hline
        Overall Accuracy & $\geq$95\% on test dataset & Yes & All models exceed 95\% (e.g., CNN: 98.36\%). \\
        \hline
        Training Time & $\leq$ 5 hours on M2 MacBook Air & Yes & Training times are within 8-95 minutes. \\
        \hline
        Generalization & Robust performance on unseen ECG signals & Partially & Some overfitting observed; further validation on external data is needed. \\
        \hline
        Performance Metrics & Evaluate using accuracy, precision, recall, F1-score & Yes & Overall metrics are reported; however, per-class performance needs enhancement. \\
        \hline
    \end{tabular}
\end{table}

\subsection{Future Work}
Based on our evaluation, we propose additional specifications for future work to address the shortcomings and further improve model performance:

\begin{enumerate}
    \item \textbf{Per-Class Recall Target:} Increase the recall (sensitivity) for clinically critical arrhythmia classes to at least 90\%.
    \item \textbf{Per-Class Precision Target:} Improve the precision for arrhythmia classes (especially classes 1 and 3) to a target of 90\% or higher.
    \item \textbf{Robust Generalization:} Validate model performance on additional external ECG datasets to ensure robustness across different populations and recording conditions.
\end{enumerate}

By setting additional targets for per-class performance, future iterations should aim to not only meet but exceed the clinical requirements for reliable and robust ECG classification.
