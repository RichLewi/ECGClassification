\documentclass[conference]{IEEEtran}
\IEEEoverridecommandlockouts
% The preceding line is only needed to identify funding in the first footnote. If that is unneeded, please comment it out.
\usepackage{cite}
\usepackage{amsmath,amssymb,amsfonts}
\usepackage{algorithmic}
\usepackage{graphicx}
\usepackage{textcomp}
\usepackage{xcolor}
\def\BibTeX{{\rm B\kern-.05em{\sc i\kern-.025em b}\kern-.08em
    T\kern-.1667em\lower.7ex\hbox{E}\kern-.125emX}}
\begin{document}

\title{ECG Classification}
\author{\IEEEauthorblockN{
    %\IEEEauthorblockN{1\textsuperscript{st} Lewis G. Richter}
    \IEEEauthorblockN{Lewis G. Richter}
    \IEEEauthorblockA{\textit{AI \& Automation} \\
    \textit{University West}\\
    Trollhättan, Sweden \\
    lewis.richter@student.hv.se}
}}

\maketitle

\begin{abstract}
Insert abstract here.
\end{abstract}

\begin{IEEEkeywords}
Insert keywords here.
\end{IEEEkeywords}

\section{Introduction}
Insert introduction here.

\section{Methodology}
Insert methodology here.

\section{Results}
Insert results here.

\section{Conclusion}
Insert conclusion here.

\section*{Acknowledgment}
Insert acknowledgment here.

\bibliographystyle{IEEEtran}
\bibliography{Insert your bibliography file here}

\begin{thebibliography}{00}
    \bibitem{b1} G. Eason, B. Noble, and I. N. Sneddon, ``On certain integrals of Lipschitz-Hankel type involving products of Bessel functions,'' Phil. Trans. Roy. Soc. London, vol. A247, pp. 529--551, April 1955.
\end{thebibliography}

\end{document}